%%%%%%%%%%%%%%%%%%%%%%%%%%%%%%%%%%%%%%%%%
% a0poster Portrait Poster
% LaTeX Template
% Version 1.0 (22/06/13)
%
% The a0poster class was created by:
% Gerlinde Kettl and Matthias Weiser (tex@kettl.de)
% 
% This template has been downloaded from:
% http://www.LaTeXTemplates.com
%
% License:
% CC BY-NC-SA 3.0 (http://creativecommons.org/licenses/by-nc-sa/3.0/)
%
%%%%%%%%%%%%%%%%%%%%%%%%%%%%%%%%%%%%%%%%%

%----------------------------------------------------------------------------------------
%	PACKAGES AND OTHER DOCUMENT CONFIGURATIONS
%----------------------------------------------------------------------------------------

\documentclass[a0,portrait]{a0poster}
\usepackage[utf8]{inputenc}
\usepackage[francais]{babel}
\usepackage[T1]{fontenc}

\usepackage{multicol} % This is so we can have multiple columns of text side-by-side
\columnsep=100pt % This is the amount of white space between the columns in the poster
\columnseprule=3pt % This is the thickness of the black line between the columns in the poster

\usepackage[svgnames]{xcolor} % Specify colors by their 'svgnames', for a full list of all colors available see here: http://www.latextemplates.com/svgnames-colors

\usepackage{times} % Use the times font

\usepackage{graphicx} 
\usepackage{booktabs} % Top and bottom rules for table
\usepackage[font=large,labelfont=bf]{caption} % Required for specifying captions to tables and figures
\usepackage{amsfonts, amsmath, amsthm, amssymb} 
\usepackage{wrapfig} % Allows wrapping text around tables and figures
\usepackage[most]{tcolorbox}
\usepackage{tikz}
\usetikzlibrary{tikzmark}
% \renewcommand\familydefault{cmss} 

\newcommand{\captioncolor}{\color{black}}
\newcommand{\equtext}[1]{\mbox{\small{#1}}}
% \makeatletter %only needed in preamble
% \renewcommand\veryHuge{\@setfontsize\veryHuge{1000pt}{1000}}
% \renewcommand\Huge{\@setfontsize\Huge{500pt}{500}}
% \makeatother
\newcommand{\subtitle}[1]{\Large{\textbf{#1}}}
\setlength{\parskip}{0em}

\begin{document}
\large
\sffamily 
%----------------------------------------------------------------------------------------
%	POSTER HEADER 
%----------------------------------------------------------------------------------------

% The header is divided into two boxes:
% The first is 75% wide and houses the title, subtitle, names, university/organization and contact information
% The second is 25% wide and houses a logo for your university/organization or a photo of you
% The widths of these boxes can be easily edited to accommodate your content as you see fit

\begin{minipage}[b]{0.75\linewidth}
\veryHuge \color{NavyBlue} \textbf{Stage Recherche en Informatique (CS 2A)} \color{Black}\\
\Huge\textit{ Génération de signaux micro-Doppler par réseaux de \\neurones }\\[18mm]
\huge \textbf{Paul LE GRAND DES CLOIZEAUX}\\[0.0cm] 
\huge LRI, CentraleSupélec, Université Paris-Saclay
\end{minipage}
%
\begin{minipage}[b]{0.23\linewidth}
\begin{flushright}
\includegraphics[width=0.7\textwidth]{logo_lri.jpg}
\end{flushright}
\end{minipage}

%\vspace{1cm} % A bit of extra whitespace between the header and poster content

%----------------------------------------------------------------------------------------

\begin{multicols}{2} % This is how many columns your poster will be broken into, a portrait poster is generally split into 2 columns

%----------------------------------------------------------------------------------------
%	CONTEXTE
%----------------------------------------------------------------------------------------

\begin{tcolorbox}[colback=blue!5!lime,colframe=green!75!black,title={\section*{Contexte}}]
\textbf{\Large{Objectif: Classifier des profils micro-Doppler d'objets volants}}\\
(profils micro-Doppler de drones et d'oiseaux fournis par l'ONERA)
\begin{center}
    \includegraphics[width=0.5\textwidth]{./Phantom_version1.jpg}
    \captionof{figure}{Exemple de drone mesuré} 
\end{center}
\end{tcolorbox}
\bigskip

%----------------------------------------------------------------------------------------
%	Profils micro-Doppler
%----------------------------------------------------------------------------------------

\begin{tcolorbox}[colback=blue!5!white,colframe=blue!75!black,title={\section*{Profils micro-Doppler}}]
\textbf{Format: Spectrogramme en temps long}
\begin{center}
    \includegraphics[width=0.7\textwidth]{./QD41.png}
    \captionof{figure}{\captioncolor Spectrogramme en temps long d'un drone}
    \includegraphics[width=0.7\textwidth]{./QD41_zoom.png}
    \captionof{figure}{\captioncolor Zoom - décalages de fréquences dus aux rotors}
\end{center}
\end{tcolorbox}
\bigskip

%----------------------------------------------------------------------------------------
%	Qu'est ce qu'un GAN?
%----------------------------------------------------------------------------------------
\begin{tcolorbox}[colback=blue!5!white,colframe=blue!75!black,title={\section*{Quantité de donnés insuffisantes}}]
\subtitle{Problèmes}
\begin{itemize}
    \item Nombre faible de profils. \textbf{Quantité}
    \item Profils hautement corrélés. \textbf{Diversité}
\end{itemize}
\textbf{\Large{Solution proposée}}\\
\textit{\Large{Data augmentation}} par génération de profils micro-Doppler artificiels par réseaux de neurones (\textbf{GAN}).
\end{tcolorbox}
\bigskip

%----------------------------------------------------------------------------------------
%	Qu'est ce qu'un GAN?
%----------------------------------------------------------------------------------------

\begin{tcolorbox}[colback=blue!5!white,colframe=blue!75!black,title={\section*{Qu'est-ce qu'un GAN (Generative Adversarial Network)?}}]
GAN : un réseau de neurones pour générer des données.
\begin{center}
    \includegraphics[width=0.7\textwidth]{./gan_schema}
    \captionof{figure}{\captioncolor Schéma d'un GAN basique} 
\end{center}
\subtitle{Fonctionnement}

Recherche d'un équilibre entre le \textbf{Générateur} et le \textbf{Discriminateur}.
\begin{itemize}
    \item \textbf{Générateur} essaye de tromper le \textbf{Discriminateur}
    \item \textbf{Discriminateur} essaye de démasquer le \textbf{Générateur}
\end{itemize}
Amélioration du \textbf{Générateur} par rétropropagation de l'erreur du \textbf{Discriminateur} sur l'image.
\end{tcolorbox}
%\bigskip

%----------------------------------------------------------------------------------------
%	Différents types de GAN
%----------------------------------------------------------------------------------------

\begin{tcolorbox}[colback=blue!5!white,colframe=blue!75!black,title={\section*{Types de GANs}}]
\subtitle{Architectures de réseau}\\
D discriminateur, G générateur, c labels, x données réelles, z vecteur de bruit
\begin{center}
\begin{minipage}{0.24\textwidth}
    \includegraphics[width=1.0\textwidth]{./GAN_normal.png}
\end{minipage}
\begin{minipage}{0.31\textwidth}
    \includegraphics[width=1.0\textwidth]{./CGAN_structure.png}
\end{minipage}
\begin{minipage}{0.35\textwidth}
    \includegraphics[width=1.0\textwidth]{./INFOGAN_structure.png}
\end{minipage}
\end{center}
\subtitle{Fonctions d'erreur du Discriminateur}
\begin{itemize}
    \item Entropie croisée
    \item Distance de Wasserstein
\end{itemize}
\end{tcolorbox}
\bigskip
%----------------------------------------------------------------------------------------
%	Mesurer les performances d'un GAN?
%----------------------------------------------------------------------------------------

\begin{tcolorbox}[colback=blue!5!lime,colframe=green!75!black,title={\section*{Mesurer les performances d'un GAN?}}]
\begin{itemize}
    \item Pas possible d'évaluer le réseau sur une base de test.
    \item Évaluation humaine peu fiable, notamment pour les profils micro-Doppler.
\end{itemize}
\end{tcolorbox}
\bigskip

%----------------------------------------------------------------------------------------
%	FID
%----------------------------------------------------------------------------------------

\begin{tcolorbox}[colback=blue!5!white,colframe=blue!75!black,title={\section*{FID (Fréchet Inception Distance)}}]
\textbf{Evaluation d'une distance entre deux ensembles d'images}\\
\textbf{InceptionV3}, réseau de neurones à convolution entraîné sur \textit{ImageNet} est utilisé pour extraire des motifs de l'image.\\
Comparaison des statistiques de sorti l'avant-dernière couche du réseau.%
\Large{
\[X_{\equtext{real}}=\mathcal{N}(\mu_{\equtext{real}}, \Sigma_{\equtext{real}}), X_{\equtext{generated}}=\mathcal{N}(\mu_{\equtext{generated}}, \Sigma_{\equtext{generated}})\]
\[FID = ||\mu_{\equtext{real}} - \mu_{\equtext{generated}}||^2 + Tr(\Sigma_{\equtext{real}} + \Sigma_{\equtext{generated}}  - (\Sigma_{\equtext{real}} \Sigma_{\equtext{generated}})^{1/2})\]
}
\end{tcolorbox}
\bigskip

%----------------------------------------------------------------------------------------
%	Résultats
%----------------------------------------------------------------------------------------

\begin{tcolorbox}[colback=blue!5!white,colframe=blue!75!black,title,title={\section*{Résultats}}]
\newcommand{\centeredmark}[1]{\centerline{\tikzmark{#1}}}
\begin{minipage}{0.2\textwidth}
    \includegraphics[width=1.0\textwidth]{./iter=20000.png}
    \centeredmark{a}
\end{minipage}%
\begin{minipage}{0.2\textwidth}
    \includegraphics[width=1.0\textwidth]{./iter=40000.png}
    \centeredmark{b}
\end{minipage}%
\begin{minipage}{0.2\textwidth}
    \includegraphics[width=1.0\textwidth]{./iter=60000.png}
    \centeredmark{c}
\end{minipage}%
\begin{minipage}{0.2\textwidth}
    \includegraphics[width=1.0\textwidth]{./iter=80000.png}
    \centeredmark{d}
\end{minipage}%
\begin{minipage}{0.2\textwidth}
    \includegraphics[width=1.0\textwidth]{./iter=100000.png}
    \centeredmark{e}
\end{minipage}%
\begin{center}
\begin{tikzpicture}[remember picture]
    \node (image) at (0,0) {\includegraphics[width=0.9\textwidth]{./fid_per_iterations.png}};
    %\draw[help lines,xstep=1.0,ystep=1.0] (-15,-15) grid (15,15);
    \foreach \i/\j in {(-8.8,3.8)/a, (-2.8, -3)/b, (3, -2.5)/c, (9, -6)/d, (15, 7.5)/e} \draw [red,->, line width=5pt] ({pic cs:\j}) -- \i;
\end{tikzpicture}
\end{center}
\begin{center}
\captionof{figure}{FID au cour des itérations avec profils correspondants}
\end{center}
\end{tcolorbox}
\bigskip

%----------------------------------------------------------------------------------------
%	PERSPECTIVES
%----------------------------------------------------------------------------------------

\begin{tcolorbox}[colback=red!5!orange,colframe=red!75!black,title={\section*{Perspectives}}]
Les signaux générés ne sont pas de très bonne qualité.\\
\textbf{Améliorations}
\begin{itemize}
    \item \textit{Parameters tunning}
    \item GANs \textit{image-to-image}, tel que \textbf{CycleGAN}, à partir de profils micro-Doppler simulés permettant de générer des profils de drones absents de la base de données.
    \item GANs plus avancés (par exemple \textbf{StyleGAN})
\end{itemize}
\end{tcolorbox}

\end{multicols}
\end{document}
